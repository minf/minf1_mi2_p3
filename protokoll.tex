\documentclass[a4paper,10pt]{article}

\usepackage{geometry}
%\geometry{a4paper,left=2.5cm,right=2cm,top=3cm,bottom=2cm}

\usepackage[utf8x]{inputenc}
\usepackage[bookmarks,colorlinks=false,pdfborder={0 0 0}]{hyperref}
\hypersetup{pdftitle={2. Praktikum: Modellierung von Informationssystemen}}
\usepackage{url}
\usepackage[ngerman]{babel}
\usepackage{graphicx}

\parindent 0pt 
\parskip 10pt

\title{3. Praktikum: Modellierung von Informationssystemen}
\author{Andreas Krohn \and Benjamin Vetter \and ..bitte ausfüllen..}

\begin{document}

\maketitle

\tableofcontents

\section{Analyse}
\emph{Welcher konkreter Konfigurator soll re-implementiert werden?}

http://carconfig.toyota-europe.com/

\emph{Welche Schwachstellen sollen in der neuen Fassung vermieden werden?}

\begin{itemize}
 \item Auswahl des Modells soll in dem Konfigurator selbst möglich sein
 \item Wizard - Schrittweises Konfigurieren des Autos
 \item Menü rechts oben.. Sprechendere Namen, nähere Angaben $\rightarrow$ Pro Option eine Seite im Wizard
\end{itemize}

\emph{Welche Fahrzeuggrundtypen gibt?}

\begin{tabular}{|l|}
\hline
Name \\
\hline
iQ \\
Aygo \\
Yaris \\
Urban Cruiser \\
Auris\\
\hline
\end{tabular}

\emph{Was ist konfigurierbar?}

\emph{Welche Komponenten sind miteinander verbaubar?}



\section{sonstiges\ldots}
\begin{itemize}
 \item Modelldatenbank
 \item Teile und Konfigurationsoptionen (Farbe, etc..) in DB
 \item Kombinierbarkeit/Konfigurierbarkeit/Regeln in XML-Dateien (die dann Modelle/Teile.. referenzieren)
 \item Verbaubarkeitsregeln in gesondertem Editor?
 \item Ausblick: Workflow/Ablauf konfigurierbar
\end{itemize}

\end{document}

