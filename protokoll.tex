\documentclass[a4paper,10pt]{article}

\usepackage{geometry}
%\geometry{a4paper,left=2.5cm,right=2cm,top=3cm,bottom=2cm}

\usepackage[utf8x]{inputenc}
\usepackage[bookmarks,colorlinks=false,pdfborder={0 0 0}]{hyperref}
\hypersetup{pdftitle={2. Praktikum: Modellierung von Informationssystemen}}
\usepackage{url}
\usepackage[ngerman]{babel}
\usepackage{graphicx}

\parindent 0pt 
\parskip 10pt

\title{3. Praktikum: Modellierung von Informationssystemen}
\author{Andreas Krohn \and Benjamin Vetter \and Erik Andresen \and Jan Depke}

\begin{document}

\maketitle

\tableofcontents

\section{Analyse}
\emph{Welcher konkreter Konfigurator soll re-implementiert werden?}

http://carconfig.toyota-europe.com/

\emph{Welche Schwachstellen sollen in der neuen Fassung vermieden werden?}

\begin{itemize}
 \item Auswahl des Modells soll in dem Konfigurator selbst möglich sein
 \item Wizard - Schrittweises Konfigurieren des Autos
 \item Menü rechts oben.. Sprechendere Namen, nähere Angaben $\rightarrow$ Pro Option eine Seite im Wizard
\end{itemize}

\emph{Welche Fahrzeuggrundtypen gibt?}

\begin{tabular}{|l|}
\hline
Name \\
\hline
iQ \\
AYGO \\
Yaris \\
Urban Cruiser \\
Auris \\
Verso \\
Avensis \\
RAV4 \\
Prius \\
Land Cruiser \\
Land Cruiser V8 \\
\hline
\end{tabular}

\emph{Was ist konfigurierbar?}

\emph{Welche Komponenten sind miteinander verbaubar?}

\section{Design}
Klassenmodell
\begin{itemize}
 \item Es gibt eine Klasse \emph{Option} (mögl. Ausprägungen: Chassis "`Yaris"', Michelin Reifen "`XYZ"', Metallic-Lackierung "`Schwarz"'\ldots)
 \item Eine Option hat einen Identifier und gehört zu einer \emph{Kategorie} (z.B. Radio, Dachfenster, Bereifung)
 \item Pro Option gibt es eine Liste von Identifiern, mit denen sie kombinierbar ist.
\end{itemize}


Architekturmodell

Arbeitspakete und Zuständigkeiten

\subsection*{Regelsystem}

\begin{itemize}
 \item Es gibt eine Sequenz von Kategorien (Chassis $\rightarrow$ Reifen $\rightarrow$ Lack $\rightarrow$ \ldots)
 \item Es gibt eine Menge aktuell gewählter Optionen (und - implizit? - abgearbeteter Kategorien), die aktuelle \emph{Konfiguration}
 \item Zu einer Konfiguration liefert das Regelwerk eine Menge noch verfügbarer Kategorien sowie jeweils wählbarer Optionen.
\end{itemize}




\subsection*{DB}

\subsection*{GUI}

\section{Implementierung}
Programmiersprache und Realisierungsframework frei wählbar
\section{Tests}
Testpläne und -protokolle



\section{Komponenten/Schnittstellen}

\subsection*{GUI}


\subsection*{Regelengine}

Aktion:
 

Parameter:
 Liste von (bereits gewählten) Optionen

Rückgabe:
 Liste von wählbaren Optionen

\subsection*{Datenbank}


\section{sonstiges\ldots}
\begin{itemize}
 \item Modelldatenbank
 \item Teile und Konfigurationsoptionen (Farbe, etc..) in DB
 \item Kombinierbarkeit/Konfigurierbarkeit/Regeln in XML-Dateien (die dann Modelle/Teile.. referenzieren)
 \item Verbaubarkeitsregeln in gesondertem Editor?
 \item Ausblick: Workflow/Ablauf konfigurierbar
\end{itemize}


\end{document}

% vim: fileencoding=utf8
